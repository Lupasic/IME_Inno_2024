% \documentclass[aspectratio=169,notes]{beamer}
\documentclass[aspectratio=169]{beamer}
\usetheme[faculty=phil]{fibeamer}
\usepackage{polyglossia}
\setmainlanguage{english} %% main locale instead of `english`, you
%% can typeset the presentation in either Czech or Slovak,
%% respectively.
\setotherlanguages{russian} %% The additional keys allow
%%
%%   \begin{otherlanguage}{czech}   ... \end{otherlanguage}
%%   \begin{otherlanguage}{slovak}  ... \end{otherlanguage}
%%
%% These macros specify information about the presentation
\title[IME]{Introduction to Mechanical Engineering, HW CAE DYN 1} %% that will be typeset on the
\subtitle{Inverse Dynamics Problem
\\ Rifle simulation  \\ \ 
         } %% title page.
\author{Oleg Bulichev}
%% These additional packages are used within the document:
\usepackage{ragged2e}  % `\justifying` text
\usepackage{booktabs}  % Tables
\usepackage{tabularx}
\usepackage{tikz}      % Diagrams
\usetikzlibrary{calc, shapes, backgrounds}
\usepackage{amsmath, amssymb}
\usepackage{url}       % `\url`s
\usepackage{listings}  % Code listings
% \usepackage{subfigure}
\usepackage{floatrow}
\usepackage{subcaption}
\usepackage{mathtools}
\usepackage{todonotes}
\usepackage{fontspec}
\usepackage{multicol}
\usepackage{pdfpages}
\usepackage{wrapfig}
\usepackage{animate}
\usepackage{booktabs}
\usepackage{multirow}
% \usepackage{graphicx}
\usepackage{colortbl}

\graphicspath{{resources/}}
\frenchspacing

\setbeamertemplate{caption}[numbered]
\usetikzlibrary{graphs}

% \usepackage[backend=biber,style=ieee,autocite=footnote]{biblatex}
% \addbibresource{biblio.bib}
% \DefineBibliographyStrings{english}{%
%   bibliography = {References},}

\newcommand{\oleg}[2][] {\todo[color=red, #1] {OLEG:\\ #2}}
\newcommand{\fbckg}[1]{\usebackgroundtemplate{\includegraphics[width=\paperwidth]{#1}}}%frame background

\newcommand\pic[1]{(\cref{#1})}

\usepackage[framemethod=TikZ]{mdframed}
\newcommand{\dbox}[1]{
\begin{mdframed}[roundcorner=3pt, backgroundcolor=yellow, linewidth=0]
\vspace{1mm}
{#1}
\vspace{1mm}
\end{mdframed}
}

\begin{document}
\setlength{\abovedisplayskip}{0pt}
\setlength{\belowdisplayskip}{0pt}
\setlength{\abovedisplayshortskip}{0pt}
\setlength{\belowdisplayshortskip}{0pt}

\fbckg{fibeamer/figs/title_page.png}
\frame[c]{\setcounter{framenumber}{0}
    \usebeamerfont{title}%
    \usebeamercolor[fg]{title}%
    \begin{minipage}[b][6.5\baselineskip][b]{\textwidth}%
        \textcolor{black}{\raggedright\inserttitle}
    \end{minipage}
    % \vskip-1.5\baselineskip

    \usebeamerfont{subtitle}%
    \usebeamercolor[fg]{framesubtitle}%
    \begin{minipage}[b][3\baselineskip][b]{\textwidth}
        \raggedright%
        \insertsubtitle%
    \end{minipage}
    \vskip.25\baselineskip
}
%   \frame[c]{\maketitle}

\fbckg{fibeamer/figs/common.png}

\note{\scriptsize \begin{itemize}
        \item \
    \end{itemize}}

\note{
    \
}

\begin{frame}[t]{Task 1}
    \framesubtitle{Short Task Description}
    \textbf{Description}: Solve Inverse Dynamics problem for four link bar mechanism by NX Motion Analysis application.

    \textbf{Artifacts}:
    \begin{itemize}
        \item Zip archive with NX detail files (.prt) and simulation (.sim)
        \item A picture with mechanism angle limits, represented as a pie chart.
    \end{itemize}
\end{frame}

\begin{frame}[t]{Task 1}
    \framesubtitle{Extended Task Description}
    \vspace{-0.6cm}
    \begin{columns}[T,onlytextwidth]
        \begin{column}{0.59\textwidth}
            \scriptsize
            \textbf{Zip archive, which contains all needed data}: \textit{HWs/HW\_CAE\_DYN1/task\_data}

            1st joint is controllable, others --- not.
            \vspace{-0.1cm}
            \begin{enumerate}
                \item \textbf{Find angle limits} (where the mechanism stuck) for controllable joint, using NX (either Modeling, or Animation Designer).
                \item \textbf{Make the scene in Motion Analysis}. All links are made from <<Bronze>>. You need to add joints, contacts, direct earth gravity correctly.
                \item Choose the biggest angle gap between joint limits and put your link in the beginning of it.
                \item Apply constant angular acceleration for 1st joint --- 0.2 $rad / s^2$
                \item \textbf{Find a torque for 1st joint} for such angle gap, using NX (any solver):
            \end{enumerate}
        \end{column}
        \begin{column}{0.39\textwidth}
            \vspace{1cm}
            \begin{figure}[H]
                \centering\includegraphics[height=6cm,width=1\textwidth,keepaspectratio]{task_descr.png}
                % \caption{caption_name}
                \label{fig:task_descr.png}
            \end{figure}
        \end{column}
    \end{columns}
\end{frame}

\begin{frame}[t]{Task 1}
    \framesubtitle{Torque plot, which should be received}
    \vspace{-0.6cm}
    \begin{figure}[H]
        \centering\includegraphics[height=6cm,width=1\textwidth,keepaspectratio]{resources/1_sol.png}
        \label{fig:resources/1_sol.png}
    \end{figure}
\end{frame}

\begin{frame}[t]{Task 2}
    \framesubtitle{Short Task Description}
    \textbf{Description}: Create a simplified geometric model of the bullet in the barrel of the rifle at the moment of firing by NX Motion Analysis application.

    \textbf{Artifacts}:
    \begin{itemize}
        \item Zip archive with NX detail files (.prt) and simulation (.sim)
    \end{itemize}
\end{frame}

\begin{frame}[t]{Task 2}
    \framesubtitle{Extended Task Description}
    \vspace{-0.6cm}
    \begin{columns}[T,onlytextwidth]
        \begin{column}{0.59\textwidth}
            \scriptsize
            Create a simplified geometric model of the bullet in the barrel of the rifle at the moment of firing. Approximate parameters of the model:
            \begin{itemize}
                \item Bullet weight: 9 g.
                \item Barrel weight: 250 g.
                \item Gunpowder combustion force: 500 N.
                \item Powder combustion time: 0.01 s.
            \end{itemize}
            You need to:
            \vspace{-0.1cm}
            \begin{itemize}
                \item Give graphs of the velocity of the bullet and the barrel (plot (a)).
                \item Determine the velocity of the bullet at the moment of departure from the barrel.
                \item Calculate (experimentally) the spring stiffness , which should bring the striker to its initial state in about 0.02 seconds after the shot. Give a graph of the movement of the striker at the moment of firing (plot (b)). Determine the "recoil distance" of the striker.
            \end{itemize}
        \end{column}
        \begin{column}{0.39\textwidth}
            \vspace{1cm}
            \begin{figure}[H]
                \centering\includegraphics[height=6cm,width=1\textwidth,keepaspectratio]{resources/var1_0.jpeg}
                % \caption{caption_name}
                \label{fig:var1_0.jpeg}
            \end{figure}
        \end{column}
    \end{columns}
\end{frame}

\begin{frame}[t]{Task 2}
    \framesubtitle{Referenced plots}
    \vspace{-0.6cm}
    \begin{figure}[H]
        \begin{subfigure}{0.49\textwidth}
            \centering\includegraphics[height=5cm,width=1\textwidth,keepaspectratio]{resources/var1_1.jpeg}
            \caption{Graphs of the velocity of the bullet and the barrel}
            \label{fig:var1_1.jpeg}
        \end{subfigure}
        \begin{subfigure}{0.49\textwidth}
            \centering\includegraphics[height=5cm,width=1\textwidth,keepaspectratio]{resources/var1_2.jpeg}
            \caption{Graph of the movement of the striker at the moment of firing}
            \label{fig:var1_2.jpeg}
        \end{subfigure}
    \end{figure}
\end{frame}

\fbckg{fibeamer/figs/last_page.png}
\frame[plain]{}

\end{document}